\documentclass{article}
\usepackage[margin=2cm]{geometry}
\usepackage{url}
\usepackage{color}
\definecolor{gray}{rgb}{0.7,0.7,0.7}

\title{PyKurs \\ \normalsize Aufgaben, Quellen, Weiterer Verlauf, etc.}
\author{Anton Obersteiner}

\begin{document}
\maketitle
\paragraph{Quellen}
	Das Material findet sich unter \url{https://github.com/AntonObersteiner/python-tasks/}, diese Notiz in \url{/tasks/readme.pdf}

\paragraph{Struktur der Aufgaben}
	Die Aufgaben beginnen mit einer \textbf{Beschreibung}, dann kommt etwas \textbf{unvollständiger Code} und danach meine \textbf{Tests} (Um euch zu sagen, ob die Aufgabe gelöst wurde). Fügt gern eigenen Code ein, mit dem ihr eure Funktionen aufruft um zu sehen, ob sie tun, was sie sollen.
\paragraph{Lösungen}
	Zu einigen Dateien (z.B. \url{/list/find.py} gibt es Lösungen: \url{/list/_find.py}). Wenn man gar nicht weiterkommt, kann man da reinschauen, aber eigentlich sind nachbarn, Kursleitende und das Glossar die besseren Quellen.

\paragraph{Glossar}
	kurze Zusammenfassung mit Beispielen einiger der bisher besprochenen Themen: \url{https://github.com/antonobersteiner/python-lessons/blob/master/latex/slides/build/glossar.pdf}

	etwas umfangreicheres Cheatsheet: \url{https://github.com/antonobersteiner/python-cheatsheet/blob/master/cheatsheet.pdf}

\paragraph{Empfohlene Reihenfolge der Übungen} .\\
	\begin{tabular}{l|l}
		/lists/find.py \\
		/lists/sort.py \\
		/lists/multiples.py & Voraussetzung für \texttt{primes.py} \\
		/lists/marks.py & Daten \\
		/func/recursion.py & Grundlagen Funktionen \\
		/class/Mensch.py & einfache Objekte \\
		/class/Car.py & lustige Objekte, Lösung unvollständig \\
		/the\_turtle.py & macht damit, was ihr wollt \\
		/dict/calc.py & recht freie Aufgabenstellung \\
		/dict/gene\_expr.py & Datenanalyse \\
		/class/Coral.py & visuell, Objekte \\
		/func/tree.py & Rekursion mit der Turtle \\
		/lists/primes.py & mathematisch etwas interessanter \\
		/class/Vector.py & Objekte, interne Methoden \\
		/class/Planet.py & Objekte, visuell \\
		/measure/... & Vorbereitung auf's Robolab \\
	\end{tabular}

\newpage
\paragraph{Themenwünsche}
	Wer ein bestimmtes Thema näher beleuchtet haben möchte, sagt Bescheid. Vorschläge: \\
	\begin{tabular}{ll}
		Datenanalyse/-Visualisierung (matplotlib) &
		Bildverarbeitung (PIL) \\
		Web-Zeug &
		mit Betriebssystem reden? \\
		PyGame (wahrscheinlich sehr advanced) &
		Machine Learning (auch eher advanced) \\
		Nicht Python-spezifisch: & Git, Regex, \LaTeX
	\end{tabular}

\end{document}
